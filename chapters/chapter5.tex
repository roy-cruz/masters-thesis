\chapter{Novel Trigger Efficiency }

In \cite{cmscollaborationSearchNewPhysics2024}, the triggers used for data collection was the $H_T$ trigger with a threshold of $1050\;\text{GeV}$\footnote{For data from 2016, a trigger that required at least one jet with $p_\text{T} > 450\;\text{GeV}$ or $H_T > 900\;\text{GeV}$ was used due to an issue where jets were being dropped from the $H_T$ sum if they reached the energy saturation of the L1 trigger. Because the data simulated correspond to the data collection conditions in 2022, this alternative trigger was not considered.}. As a starting point, the efficiency of this previously used trigger was obtained for the EMJ simulated parameter space. The efficiency of trigger A is defined as

\begin{equation}
    \varepsilon(A) = \frac{N_{\text{A}}}{N_\text{T}},
\end{equation}

\noindent where $N_{\text{A}}$ are the number of events that pass the selection of trigger A, while $N_\text{T}$ corresponds to the total number of events. The uncertainty of these efficiencies was computed using the Clopper-Pearson interval. Figure \ref{fig:ht1050} shows the raw trigger efficiency maps for the aforementioned trigger for the models with $X_{\text{dark}}$ and $Z'$ mediators for $m_{\pi_{\text{d}}} = 20 \;\text{GeV}$.

\begin{figure}[h]
    \centering
    \begin{subfigure}{0.45\textwidth}
        \includesvg[width=\linewidth]{images/HLT/trigeffplots2D_HLT_efftype-trig_mDark-20_t-channel_HLT_PFHT1050_effs}
        \caption{t-channel}
    \end{subfigure}
    \hfill
    \begin{subfigure}{0.45\textwidth}
        \includesvg[width=\linewidth]{images/HLT/trigeffplots2D_HLT_efftype-trig_mDark-20_s-channel_HLT_PFHT1050_effs}
        \caption{s-channel}
    \end{subfigure}
    \caption{Trigger efficiency of \texttt{HLT\_PFHT1050} for the simulated parameter space.}
    \label{fig:ht1050}
\end{figure}

It is readily apparent that the performance of this trigger for shorter and longer dark pion lifetimes is comparable in the case of the $X_{\text{dark}}$ mediator. This is to be expected, as the final state has a high jet multiplicity and, even for models with higher $c\tau$, two of the four jets are QCD jets, so the event can easily be triggered on. On the other hand, for the $Z'$ model, while the efficiency is high towards a lower dark pion proper decay length and higher mediator mass, for the rest of the parameter space, the efficiency drops considerably. This is likely to be due in part to the fact that a higher portion of the jet energy is deposited outside the calorimeters for higher lifetimes. Similar observations made with the \texttt{HLT\_PFHT1050} trigger can be made for its L1 seed, the \texttt{L1\_HTT450er}, the efficiency heatmaps for which can be seen in Figure \ref{fig:htt450er}.

\begin{figure}[h]
    \centering
    \begin{subfigure}{0.45\textwidth}
        \includesvg[width=\linewidth]{images/L1/trigeffplots2D_L1_efftype-trig_mDark-20_t-channel_L1_HTT450er_effs.svg}
    \end{subfigure}
    \hfill
    \begin{subfigure}{0.45\textwidth}
        \includesvg[width=\linewidth]{images/L1/trigeffplots2D_L1_efftype-trig_mDark-20_s-channel_L1_HTT450er_effs.svg}
    \end{subfigure}
    \caption{Trigger efficiency of \texttt{L1\_HTT450er} for the simulated parameter space}
    \label{fig:ht1050}
\end{figure}

These results show the potential for an improvement in the sensitivity for EMJs in a Run 3 search, particularly for the $Z'$ model. In order to understand what the set of LLP and AD triggers under study have to offer in terms of increasing sensitivity, the efficiency of the conventional triggers with the highest efficiency for each point in the simulated parameter space was used as a baseline. Conventional triggers in this context refer to the set of all triggers in the selected 2022 trigger menu that are not in the group of triggers under study or in a list of special-purpose triggers that are not relevant to these studies. Moreover, all triggers studied are unprescaled for the data-taking period the data is simulated for, and only those paths corresponding to nominal operations are included. For each trigger, the efficiency increase, or $\text{EI}$, defined as

\begin{equation}
    \text{EI}_{\text{A}B} (m_{\text{Med}}, m_{\pi_{\text{d}}},c\tau_{\pi_{\text{d}}}) = \frac{\varepsilon_{\text{A OR B}}}{\varepsilon_{\text{B}}},
\end{equation}

\noindent is used as a measure of increased sensitivity, where $\text{A}$ is the trigger under study, and $\text{B}$ is the best conventional trigger for the given set of parameter values.

The choice of L1 (Table \ref{tab:l1-trigs}) and HLT (Table \ref{tab:hlt-trigs}) triggers for this study is based on their novelty with respect to what was available for the previous search for EMJs, and with their functionality in mind. The anomaly detection triggers, allow for a model independent way of selecting anomalous events which exotic signatures. Moreover, previous studies such as \cite{} have shown that the use of this set of triggers for the search of beyond Standard Model physics can provide some increase in sensitivity. On the other hand, because the models which give rise to EMJs predict long-lived particles in the dark sector, potential was seen for triggers targeting signatures left behind by long-lived particles. Previous studies have reinforced this hypothesis \cite{} and thus motivate their inclusion in this study.

\section{Level 1 Triggers Efficiencies}

% \subsection{Trigger Description}

% In this section, a description of the set of level-1 triggers under study is provided.

\subsection{Anomaly Detection Triggers}
% \noindent\textbf{Anomaly Detection Triggers:} 

As described in detail in \cite{}, these triggers utilize unsupervised ML techniques to apply a model-independent selection of events with potentially interesting physics that would otherwise not be selected by cut-based triggers or triggers targeting particular detector signatures. However, due to resource constraints in the L1 system and the challenges of fitting a complex ML model in an FPGA, various techniques such as quantized aware training and knowledge distillation must be implemented to deploy these triggers.

The two AD triggers considered in these studies are AXOL1TL, based on a variational autoencoder, and CICADA, based on a convolutional autoencoder. They are trained on a large unbiased sample of data, most of which amounts to known QCD physics or where there was little momentum transfer, and output an anomaly score which serves as a metric of how different an output event is from the typical events the model was trained on. However, the input to AXOL1TL and CICADA differ: the former uses L1 objects, namely the $p_T$, $\phi$ and $\eta$ of 4 $e/\gamma$, 4 $\mu$ and 10 jets, as well as the $\phi$ and magnitude of the event's missing transverse energy, while the latter uses as input the calorimeter region energies. A threshold is applied to the anomaly score output, which dictates whether or not an event is selected, and which allows for the control of the rate of the trigger itself, as a higher threshold implies a less likelihood for any particular event to meet the selection criteria.



% While triggers targeting particular expected signatures produced by physics beyond the Standard Model are effective, they become a limitation when there is no strong theoretical prior. It was under this consideration that AD triggers were developed, with the goal of increasing sensitivity to signals for which there are no dedicated signatures. These triggers utilize unsupervised ML techniques to search for new physics signatures. However, because the resources available at the level-1 trigger system are heavily restricted, as described in \ref{govorkovaAutoencodersFPGAsRealtime2022}, multiple techniques, such as knowledge distillation and quantum aware training (QAT), are implemented in order to allow these models to fit in FPGAs which can then be deployed.

% The two AD triggers considered in these studies are AXOL1TL, based on a variational autoencoder, and CICADA, based on a convolutional autoencoder. The models these triggers use are trained on an unbiased random sample of events, most of which correspond to known physics or where there was little momentum transfer. AXOL1TL uses as its input the $p_T$, $\eta$, and $\phi$ of 18 level-1 reconstructed objects, namely 4 muons, 4 electrons, and 10 jets, as well as the magnitude and $\phi$ coordinate of the event's level-1 reconstruction of the MET. The loss function used during training is:

% \begin{equation}
%     \mathcal{L_\text{AXO}} = (1-\beta)\text{MSE(Output, Input)} + \beta\text{D}_{\text{KL}}(\vec \mu, \vec \sigma)
%     \label{}
% \end{equation}

% where $\text{D}_{\text{KL}}$ is the Kullback-Leibler regularization term, and $\beta\in [0,1]$ is a hyperparameter used to adjust the optimization priority of the two terms in the loss function. The deployed trigger, however, only uses the encoder section of the model, and it outputs the mean in the latent space of the quantized model as the 



% CICADA, on the other hand, uses the calorimeter region energies. Note, however, that what is actually deployed in the FPGAs are not the full trained models, but rather a reduced version of them which 


\noindent\textbf{Displaced \& Delayed Jet Triggers:}

The Phase-1 upgrade to CMS introduced precision timing information and depth segmentation to the HCAL barrel and endcap. This enabled the introduction of these sets of triggers, which take advantage of this information in order to specifically target the longer-lived regime of the parameter space of many models with long-lived particles. In particular, they select events with jets displaced with respect to the primary vertex as well as jets delayed significantly with respect to the bunch crossing. The displaced jet triggers select events in which there are HCAL trigger towers with lower energy in the lower depths, but higher in the more exterior ones. The delayed jet triggers, on the other hand, flag events with pulses arriving late.

\noindent\textbf{High Multiplicity Muon Trigger:}

This trigger targets the decay of long-lived particles that lead to a high multiplicity of hits in any chamber of the Cathode Strip Chambers (CSC).

\subsubsection{Anomaly Detection Triggers}

% s-channel 

% 

% t-channel


When computing the efficiency of AXOL1TL and CICADA for the $Z'$ model, Figure \ref{}, and the $X_{\text{dark}}$ model, it was observed that for both of these triggers it peak in efficiency for the highest mediator mass simulated and the 

\begin{figure}
    \centering
    \includesvg[width=0.5\linewidth]{images/L1/trigeffplots2D_L1_efftype-trig_mDark-20_s-channel_L1_axol1tl_v4_200_effs.svg}
    \caption{Caption}
    \label{fig:enter-label}
\end{figure}



\subsection{Trigger Efficiency Results}

\section{High Level Trigger Efficiencies}


\begin{landscape}
    \begin{table}[]
        \centering
        \begin{tabular}{llc}
            \hline
            \textbf{Trigger} & \textbf{Description} & \textbf{Rate (Hz)}\\
            \hline
            \texttt{L1\_AXO\_Nominal} (version 4) & AXOL1TL anomaly score > $25.9375$ & 200 \\
            \texttt{L1\_AXO\_Nominal} (version 3) & AXOL1TL anomaly score > $1161.25$ & 300 \\
            \texttt{L1\_CICADA\_Nominal} (version 2.1.2) & CICADA anomaly score > $121.0$ & -999\\
            \texttt{L1\_DoubleLLPJet40} & a & -999\\
            \texttt{L1\_HTT200\_SingleLLPJet60} & a & -999\\
            % \texttt{L1\_HTT240\_SingleLLPJet70} & a & -999\\
            \texttt{L1\_SingleMuShower\_Nominal} & a & -999\\
            \hline
        \end{tabular}
        \caption{Level-1 Triggers studied}
        \label{tab:l1-trigs}
    \end{table}
    \begin{table}[]
        \centering
        \begin{tabular}{llc}
            \hline
            \textbf{Trigger} & \textbf{Description} & \textbf{Rate (Hz)}\\
            \hline
            \texttt{HLT\_HT170\_L1SingleLLPJet\_DisplacedDijet40\_DisplacedTrack} & & \\
            \texttt{HLT\_HT200\_L1SingleLLPJet\_DisplacedDijet40\_DisplacedTrack} & & \\
            \texttt{HLT\_HT200\_L1SingleLLPJet\_DisplacedDijet60\_DisplacedTrack} & & \\
            \texttt{HLT\_HT270\_L1SingleLLPJet\_DisplacedDijet40\_DisplacedTrack} & & \\
            \texttt{HLT\_HT320\_L1SingleLLPJet\_DisplacedDijet60\_Inclusive} & & \\
            \texttt{HLT\_HT420\_L1SingleLLPJet\_DisplacedDijet60\_Inclusive} & & \\
            \texttt{HLT\_HT200\_L1SingleLLPJet\_DelayedJet40\_SingleDelay1nsTrackless} & & \\
            \texttt{HLT\_HT200\_L1SingleLLPJet\_DelayedJet40\_SingleDelay2nsInclusive} & & \\
            \texttt{HLT\_HT200\_L1SingleLLPJet\_DelayedJet40\_DoubleDelay0p5nsTrackless} & & \\
            \texttt{HLT\_HT200\_L1SingleLLPJet\_DelayedJet40\_DoubleDelay1nsInclusive} & & \\
            \texttt{HLT\_HT200\_L1SingleLLPJet\_DisplacedDijet40\_Inclusive1PtrkShortSig5} & & \\
            \hline
        \end{tabular}
        \caption{High-Level Triggers (HLT) studied}
        \label{tab:hlt-trigs}
    \end{table}
\end{landscape}