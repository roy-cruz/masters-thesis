\chapter{Introduction}

The Standard Model of particle physics has been phenomenally succesful in describing all observed fundamental particles and their interactions. Its predictive power has been repeatedly demonstrated, such as in 2012 with the discovery of the Higgs boson at the Large Hadron Collider (LHC) at CERN \cite{collaborationObservationNewBoson2012}, as well as with the discovery of the top quark at the Tevatron in 1955 \cite{abachiObservationTopQuark1995}. Despite its success, there exists an abundance of signs that the Standard Model is an incomplete theory. Among these, we have the matter-antimatter asymmetry in the universe \cite{}, the existence of dark matter, the neutrino's non-zero mass, the lightness of the Higgs boson \cite{saikumarExploringFrontiersChallenges2024}, and the inability to describe gravity within this framework \cite{}.

Of particular interest in this work is the elusive dark matter. Evidence for this type of matter, which does not interact with baryonic matter through any of the known forces except the gravitaional force, is abundant. Among this evidence, we have the cosmic microwave background anisotropies, graviational lensing, and the rotation curves of galaxies \cite{}. From these pieces of evidence, we have come to the conclusion that the majority of the matter in the universe is non-baryonic and not described by the Standard Model. In fact, the total energy density of all baryonic matter in the universe, $\Gamma_{\text{B}} \sim 0.05$, is much smaller than the total matter energy density in the universe, i.e. $\Gamma_{m} \sim 0.32$ \cite{}. This means that the majority of the matter in the universe, that which constitutes $\Gamma_{\text{DM}} \sim 0.27$, is non-baryonic and not described by the Standard Model.
