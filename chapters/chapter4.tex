
\chapter{Tables in \LaTeX}  

\section{How to Incorporate Tables}
\noindent Tables can be typed directly in the editor, but it's tedious, time-consuming, and can lead to errors. The easiest way to format a table to use in \LaTeX\ is to simply type it in MS Excel and convert to \LaTeX\ using \textbf{Excel2LATEX}: https://ctan.org/tex-archive/support/excel2latex. Download the zip file, unzip, run the .xla file and you're ready to go! Type your data on MS Excel, go to Add-ins, click on the Excel@LATEX icon. A window with some options will appear. Click on 'Copy to clipboard' and paste it in your \LaTeX\ document. You must enable macros for the add-in to work. Be sure to format the table as you want it to appear in your document. You can reference tables the same way as images, but using the \textbf{ref command and tab}, as seen in Table \ref{tab:table2}.
% Table generated by Excel2LaTeX from sheet 'Sheet1'
\begin{table}[htbp]
  \centering
  \caption{A centered random table}
    \begin{tabular}{ccc}
    Material & Weight (lb) & Height (ft) \\
    1     & 4.5   & 2 \\
    2     & 3.1   & 3 \\
    3     & 6.8   & 4.5 \\
    4     & 9.8   & 10 \\
    \end{tabular}%
  \label{tab:table1}%
\end{table}%https://www.overleaf.com/project/5faa3f6b47f4c254f5e0c64e

\noindent Including centering and borders/lines.

% Table generated by Excel2LaTeX from sheet 'Sheet1'
\begin{table}[htbp]
  \centering
  \caption{This looks weird}
    \begin{tabular}{|c|c|c|}
    \toprule
    \textbf{Material} & \textbf{Weight (lb)} & \textbf{Height (ft)} \\
    \midrule
    1     & 4.5   & 2 \\
    \midrule
    2     & 3.1   & 3 \\
    \midrule
    3     & 6.8   & 4.5 \\
    \midrule
    4     & 9.8   & 10 \\
    \bottomrule
    \end{tabular}%
  \label{tab:table3}%
\end{table}%

\noindent That looks bad, so sub the top, mid and bottomrule for hline.

% Table generated by Excel2LaTeX from sheet 'Sheet1'
\begin{table}[htbp]
  \centering
  \caption{This is better!}
    \begin{tabular}{|c|c|c|}
    \hline
    \textbf{Material} & \textbf{Weight (lb)} & \textbf{Height (ft)} \\
    \hline
    1     & 4.5   & 2 \\
    \hline
    2     & 3.1   & 3 \\
    \hline
    3     & 6.8   & 4.5 \\
    \hline
    4     & 9.8   & 10 \\
    \hline
    \end{tabular}%
  \label{tab:table2}%
\end{table}%


\noindent Or use thin lines instead.

% Table generated by Excel2LaTeX from sheet 'Sheet1'
\begin{table}[htbp]
  \centering
  \caption{This is best!}
    \begin{tabular}{ccc}
    \toprule
    \textbf{Material} & \textbf{Weight (lb)} & \textbf{Height (ft)} \\
    \midrule
    1     & 4.5   & 2 \\
    2     & 3.1   & 3 \\
    3     & 6.8   & 4.5 \\
    4     & 9.8   & 10 \\
    \bottomrule
    \end{tabular}%
  \label{tab:table4}%
\end{table}%







\subsection{Bigger Tables}
\noindent For tables that don't fit the width of the paper, use the \textbf{landscape command}, as shown below:

\begin{landscape}
\noindent Tables may be too big, so you can use footnotesize to make the font smaller, like in Table \ref{tab:table6}.
% Table generated by Excel2LaTeX from sheet 'Sheet1'
\begin{table}[htbp]
\footnotesize
  \centering
  \caption{You can use the footnotesize command to make the font smaller}
\begin{tabular}{ccccc}
\addlinespace
\toprule
    Temp (C) & Composition (mole\%) & Tf (C) & Heat of Fusion (kJ/kg) & Relative Cost \\
    \midrule
    367   & NaOH, NaCl (20) & 370   & 370   & 1.001 \\
          & KOH   & 360   & 167   & 4.39 \\
    347   & \ce{KNO3}, KBr (10), KCl (10) & 342   & 140   & 4.04 \\
          & NaCl, KCl (24), LiCl (43) & 346   & 281   & 5.44 \\
    328   & \ce{KNO3}  & 337   & 116   & 3.8 \\
          & \ce{KNO3}, KCl (6) & 320   & 150   & 3.33 \\
          & NaOH  & 318   & 158   & 2.78 \\
          &       & 286-299 & 3162  & 1.2 \\
    308   & NaCl,  NaOH (93.7) & 314   & -     & - \\
          & \ce{NaNO3} & 310   & 174   & 1.35 \\
          & NaF, \ce{NaNO3} (96.5) & 304   & -     & - \\
          & LiOH, KOH (60) & 314   & 341   & 3.57 \\
    289   & \ce{Na2SO4}, NaCl (8.4), \ce{NaNO3} (86.3) & 287   & 176   & 1.3 \\
          & \ce{NaNO3}, NaCl (6.4) & 284   & 171   & 1.2 \\
          & \ce{KNO3}, \ce{Ba(NO3)2} (87) & 290   & 124   & 2.85 \\
          & \ce{NaNO2} & 282   & 212   & 3.33 \\
          & \ce{NaNO2}, \ce{KNO3} (45.2) & 285   & 152   & 3.61 \\
          & NaOH, NaCl (7.8), \ce{Na2CO3} (6.4) & 282   & 316   & 2.9 \\
    \bottomrule
\end{tabular}
  \label{tab:table6}
\end{table}
\normalsize
\end{landscape}