

\chapter{Introduction}  
\section{Motivation, Purpose, Justification}

\noindent Hi! We encourage you to visit https://libguides.uprm.edu/writingclinics. The GWF Writing Clinics provide in-depth, useful information for preparing: abstracts, literature reviews, citations (Mendeley), academic writing, thesis outline, grammatically-sound writing, presentations (communication strategies in oral presentations and poster sessions) and visual design.  %Dummy text. Replace with your text.

\section{Objectives, Research Questions, Hypothesis}


\noindent Example of bullet items. Mention objectives, research questions, or hypotheses:
\begin{itemize}
\item Analyze
\item Study 
\item Identify
\item Construct
\item Analyze 
\end{itemize}
	
\noindent Example of numbered items. The work is divided in three phases: 

\begin{enumerate}
\item Collect data
\item Build model
\item Validate results
\end{enumerate} 


\section{Chapter Summary}

\noindent
This is how you add indented descriptive paragraphs. This thesis consists of 6 chapters which are briefly summarized below:
\begin{description}
\item [Chapter 1:] Discusses project justification and objectives. Serves as an introduction to the investigation. \lipsum[1][5-7]
\item [Chapter 2:] Presents an overview of \lipsum[1][5-7]
\item [Chapter 3:] Provides a means of presenting the \lipsum[1][5-7]
\item [Chapter 4:] Discusses what a \lipsum[1][5-7]
\item [Chapter 5:] Presents a simulation model for \lipsum[1][5-7]
\item [Chapter 6:] Presents Results for \lipsum[1][5-7]
\item [Chapter 7:] Makes final remarks regarding the study and proposes several ideas for future work.
\end{description}

