\chapter{Conclusions}
\label{chap:conclusions}

\section{Prospects for the Run 3 Phase of the Emerging Jets Analysis}

Efforts towards a full Run 3 data EMJ search in CMS is underway. This next phase of the analysis looks towards building upon previous iterations by extending the parameter space by considering models with longer-lived dark pions as well as models with a $Z'$ boson mediator. However, as shown in the efficiency maps in Figures \ref{fig:ht1050} and \ref{fig:htt450er}, the trigger strategy must be adequately adapted to account for the lower multiplicity of jets in EMJ production through $Z'$, as the sensitivity of the trigger previously used to these new models falters under these new circumstances. The work presented here has shown that novel AD and LLP L1 triggers not previously available could serve to increase search sensitivity not only to $Z'$ EMJs, but also to bifundamental models with higher $c\tau_{\text{dark}}$ and lower $m_{\text{Med}}$.

The novel AD triggers CICADA and AXOL1TL, trained on unbiased data mainly consisting of QCD events, were able to trigger EMJ events, particularly for shorter-lived dark pions and higher mediator masses in the bifundamental model. However, because future searches seek to expand toward models with longer-lived dark pions, in their current state, these triggers do not target the parameter space of interest. Note that because these triggers are ML-based and are being iteratively trained and fine-tuned, a future version of AXOL1TL or CICADA might be able to trigger on EMJs from models in the parameter space of interest. More studies would need to be conducted to determine whether that would be the case once those versions are available to the collaboration.

Long-lived particle triggers were incorporated into these studies, as their very design seems to promise improved sensitivity to EMJs. This was effectively seen for \texttt{L1\_HTT200\_SingleLLPJet60} and \texttt{L1\_DoubleLLPJet40}, which offered their peak increase in sensitivity in the parameter space of interest: lower mediator mass and longer-lived dark pions. However, a closer examination of the actual improvement offered leads to the conclusion that it is marginal.

In contrast to the aforementioned triggers, \texttt{L1\_SingleMuShower\_Nominal}, which targets high muon shower multiplicity in the CSC detector of the muon system, this trigger not only targets the parameter space of interest, but offers a sensitivity increase significantly larger than the other triggers studied. Given these findings, a trigger strategy, seeded by the high multiplicity muon shower trigger, could be formulated in the HLT which could aid in the search for EMJs in the next iteration of the EMJ analysis.

While work continues on the analysis in preparation for the end of Run 3, further trigger studies could be conducted for other triggers the analysis team deems to have potential to increase our sensitivity for EMJs. However, these results already give an excellent baseline of what the trigger menu has to offer in the search for EMJs. Currently, the next steps of the analysis are underway in which the analysis strategy is being established. This has manifested as studies conducted by members of the analysis team focused on how EMJs models with longer-lived dark pions deposit their energy in the calorimeters, how that energy is distributed, and how these factors could serve as discriminating variables between QCD jets and EMJs in the next phase of the analysis.

\section{Reference Run Ranking and DQM Explore}

With the age of the HL-LHC approaching, a more robust and automated DQM workflow is more necessary than ever. For this reason, the CMS collaboration is undergoing efforts to incorporate ML into the certification workflow in order to improve monitoring granularity and accuracy ahead of the end of Run 3. However, these improvements demand the development of new tools that allow for the granular exploration of anomalous LSs, as well as tools that can facilitate the retraining and upkeep of ML models that will be implemented. The work presented in Chapter \ref{chap:trackdqm} seeks to support not only these efforts, but also the efforts of the EMJ search and all other searches in CMS by helping build a more robust DQM infrastructure that can better guarantee the quality of the data used in the collaboration's discoveries.

DQM Explore, even in its early stages of development, was proven to be crucial for any efforts to integrate ML into the DQM workflow. Without proper tools for shifters and experts to investigate flagged LSs, ML efforts would be severely hampered. The utility of this toolkit was demonstrated in the case of run 380238. While experts knew there was something anomalous with some LSs of this run, the extent and severity of the issue would have been much more difficult without the interactive per-LS plot that were generated using DQM Explore. Given the proven utility of these tools, work is underway to integrate them into DIALS, a central platform that will be used by the entire collaboration.

In addition to DQM Explore, RRR is being developed to serve as another toor in the shifter's and shift leader's belt. Once fine-tuned and deployed in DIALS, it will serve as a way for reference runs to be selected in an automated way. However, once ML is implemented in the DQM workflow, this tool will be central in the automated upkeep and retraining of the deployed models. Preliminary tests have shown that it is capable of replicating experts' assessments of reference runs to some degree. However, further testing is required, particularly by shift leaders, in order to fine-tune and add features based no their feedback.