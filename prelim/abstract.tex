



%__________   ABSTRACT ENGLISH ________________________________
\vspace*{0.5in}
\begin{center}
	\section*{ABSTRACT}
\end{center}
\addcontentsline{toc}{section}{ABSTRACT} %para que aparezca en la tabla de contenido

\noindent
Previous CMS searches for emerging jets (EMJs) produced by dark pion decays into Standard Model quarks have focused on confined dark sector models with a bifundamental scalar mediator and dark pion decays primarily within the tracker volume. Additional constraints on dark pion lifetime and mass, as well as mediator mass, have been set by a search for long-lived particle (LLP) showers in the muon system. However, EMJs produced by decays within the calorimeter remain relatively unexplored, marking a gap that can be effectively addressed in Run 3. This thesis presents results for trigger efficiency studies of anomaly detection and LLP triggers to assess their potential for enhancing sensitivity to EMJs, particularly beyond the tracker, as well as the prospects of exploring models with a neutral vector boson mediator. In addition, work on the development of a data exploration and reference run ranking tools to support the integration of anomaly detection techniques into CMS's Data Quality Monitoring (DQM) infrastructure for detector monitoring and the automated flag of anomalous data is presented. These efforts seeks to improve the collection of data for physics analyses.


%____________________________________________________________





\newpage




%__________   ABSTRACT ESPANOL  ______________________________

\vspace*{0.5in}
\begin{center}
	\section*{RESUMEN}
\end{center}
\addcontentsline{toc}{section}{RESUMEN} %para que aparezca en la tabla de contenido

\noindent
Las búsquedas previas del experimento CMS de jets emergentes (\textit{emerging jets}, EMJs) producidos por desintegraciones de piones oscuros en quarks del Modelo Estándar se han centrado en modelos de sectores oscuros confinados con un mediador escalar bifundamental y desintegraciones de piones oscuros que ocurren principalmente dentro del volumen del trazador (\textit{tracker}). Se han establecido restricciones adicionales sobre la vida media y la masa del pion oscuro, así como sobre la masa del mediador, mediante una búsqueda de lluvias de partículas de larga vida (\textit{long-lived particles}, LLPs) en el sistema de muones. Sin embargo, los EMJs producidos por desintegraciones dentro del calorímetro siguen siendo relativamente inexplorados, lo que representa una oportunidad que puede abordarse eficazmente durante el \textit{Run 3}. Esta tesis presenta resultados de estudios de eficiencia de \textit{trigger} utilizando algoritmos de detección de anomalías y disparos sensibles a LLPs, con el objetivo de evaluar su potencial para mejorar la sensibilidad a los EMJs, en particular más allá del trazador, así como para explorar modelos con un mediador bosón vectorial neutro. Además, se presenta el trabajo relacionado con el desarrollo de herramientas para la exploración de datos y la jerarquización de corridas de referencia, con el fin de apoyar la integración de técnicas de detección de anomalías en la infraestructura de Monitoreo de Calidad de Datos (\textit{Data Quality Monitoring}, DQM) de CMS, para el monitoreo del detector y la señalización automatizada de datos anómalos. Estos esfuerzos buscan mejorar la recolección de datos para los análisis de física.
%____________________________________________________________
